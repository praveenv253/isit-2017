\documentclass[conference]{IEEEtran}

\usepackage[utf8]{inputenc} % allow utf-8 input
\usepackage[T1]{fontenc}    % use 8-bit T1 fonts
\usepackage{cite}
\usepackage[pdftex]{graphicx}
\usepackage{amsmath}
\usepackage{amsfonts}
\usepackage{amssymb}
\usepackage{mathtools}
\usepackage{array}
\usepackage{enumitem}
\usepackage{color}
\usepackage{url}
\usepackage[colorlinks=true,allcolors=black,urlcolor=blue]{hyperref}

\graphicspath{{images/}}
\DeclareGraphicsExtensions{.pdf,.jpeg,.png}
\interdisplaylinepenalty=2500

% Math commands for matrix and vector fonts
\providecommand{\v}{}
\renewcommand{\v}[1]{\underline{#1}}
\providecommand{\vhat}{}
\renewcommand{\vhat}[1]{\underline{\hat{#1}}}
\providecommand{\m}{}
\renewcommand{\m}[1]{{\bf #1}}
\providecommand{\j}{}
\renewcommand{\j}{\jmath}

% Custom math operators
\DeclareMathOperator*{\argmin}{arg\,min}
\DeclarePairedDelimiter\abs{\lvert}{\rvert}
\DeclarePairedDelimiter\norm{\lVert}{\rVert}
\DeclarePairedDelimiter\floor{\lfloor}{\rfloor}

\newcommand{\Phiorho}{\Phi\!\circ\!\rho}

\renewcommand*\ttdefault{lmtt}

% Theorems, lemmas, etc.
\newtheorem{theorem}{Theorem}
\newtheorem{lemma}[theorem]{Lemma}   % Shares numbering with theorem
\newtheorem{definition}{Definition}

% correct bad hyphenation here
%\hyphenation{op-tical net-works semi-conduc-tor}

\title{Minimax lower bounds for source localization}

\author{
	\IEEEauthorblockN{
		Praveen Venkatesh\IEEEauthorrefmark{1}
		and Pulkit Grover\IEEEauthorrefmark{2}
	}
	\IEEEauthorblockA{
		Electrical \& Computer Engineering,
		and the Center for the Neural Basis of Cognition,
		Carnegie Mellon University \\
		\IEEEauthorrefmark{1}\href{mailto:vpraveen@cmu.edu}{\texttt{vpraveen@cmu.edu}}
		\IEEEauthorrefmark{2}\href{mailto:pulkit@cmu.edu}{\texttt{pulkit@cmu.edu}}
	}
}

\begin{document}

\maketitle
\thispagestyle{plain}
\pagestyle{plain}

\begin{abstract}

The ``source localization'' problem is one that arises in several fields. For
instance, in neuroscience, one might wish to use electroencephalography (EEG)
measurements to locate the source of brain activity.
\textcolor{red}{<Briefly mention other applications>}.

The common underlying factor in each of these problems is the presence of a
point source, whose signal passes through a diffusive medium and is sensed by
an array of sensors. We then wish to recover the location of the source using
noisy sensed values. The diffusive medium is modeled as acting like a low-pass
filter. The noise is taken to be additive gaussian noise \textcolor{red}{which
may have a non-diagonal covariance matrix}.

\textcolor{red}{Rephrase this para.}
We give minimax lower bounds on the squared error in estimating the location of
the source, which applies to all estimators, when using unformly distributed
sensors and for certain classes of impulse response of the low-pass filter. Our
bounds are the first to give a scaling in terms of sensor density.

\end{abstract}

\section{Introduction}

\textcolor{red}{Introduction needs to be re-written.}

Electroencephalograhy (EEG) is a system used to record the electrical activity
of the brain. It is a non-invasive system, which may use anywhere from around
10 to 250 electrodes placed on the scalp, to sense electric potentials. These
potentials are generated by neuronal activity within the
brain~\cite{Buzsaki2012Origin}. These sources of neural activity -- neurons or
groups of neurons -- are usually modeled as current
dipoles~\cite{Nunez2006Electric}.

Often, it is of clinical or scientific interest to reconstruct brain activity
patterns from EEG sensor measurements. This process is what is broadly referred
to when we speak of the term ``source localization''. In this paper, I restrict
the discussion to that of localizing a \emph{single} dipole, and refer to this
process as ``dipole source localization''. Several algorithms for source
localization have been proposed over the years
(see~\cite{Baillet2001Electromagnetic} for an excellent review). However, until
recently, there has been only one work that has theoretically examined the
fundamental limits of source localization accuracy.

Mosher et.\ al.~\cite{Mosher1993Error} first propounded Cramer-Rao lower bounds
for unbiased source localization algorithms. Their work also forms the basis
for setting up the source localization problem and deriving more universally
applicable bounds in this paper. The Cramer-Rao lower bound is not in itself a
completely satisfactory answer to the question of fundamental limits for source
localization accuracy. This is because Cramer-Rao bounds only apply to unbiased
estimators (or to estimators of \emph{known} bias). In practice, however, many
popular source localization algorithms have been shown to be biased, for e.g.\
the Minimum Norm Estimate (MNE)~\cite{Hamalainen1994Interpreting} is known to
bias the solution towards the surface of the head~\cite{Lin2006Assessing}. It
is also known that unbiased estimators can have worse mean-squared-error
performance than biased estimators. So Cramer-Rao bounds are useful only to a
degree, and there may be benefit in deriving bounds that hold for all
estimators.

To this end, Grover~\cite{Grover2016Fundamental} recently proposed a new lower
bound for the error in localizing a single dipole. This bound used techniques
from information theory, for bounding the source localization error by
expressing it in terms of the mutual information across some channel, and then
bounding the mutual information by the capacity of the channel. In deriving
this bound, however, the scalp potential was never sampled; equivalently, it
was assumed that an \emph{infinite} number of sensors was available to sample
the EEG signal at every point on the scalp. Clearly, this is a poor assumption
and could result in a very loose bound, because it allows for more information
to be available at the receiver than otherwise possible. The lower bound also
relies on certain relaxations that allow the dipole to distribute its power in
non-physical ways in frequency domain. This once again makes the bound loose by
assigning the channel a higher capacity than what it could otherwise have.

To address the shortcomings of the Cramer-Rao and information theoretic lower
bounds, I apply two minimax methods to the problem of computing a lower bound
for localization error in the idealized EEG dipole source localization problem.
I briefly review two minimax methods -- Le Cam's, and Fano's local
method~\cite{Duchi2015Information} -- and describe my approach in deriving the
minimax lower bound using each of these for spherical and one-dimensional head
models.

\section{The source localization problem}
\label{sec:source-localization}

We start by giving a detailed description of the one-dimensional setting of the
source localization problem. Recall that, in this problem, we're trying to
determine the location of a point source by observing it through a diffusive
medium using a sensor array.

\subsection{Description of the domain}

We assume that the point source is located on a circle of circumference $T$ (a
circular domain enables us to use symmetry arguments to simplify the proof). We
can view this domain as a line, on which signals are periodic with period $T$.
The point source is therefore located somewhere within one period. We denote
the set of possible locations by $\Theta = [0, T)$. Due to the periodic nature
of the domain, a source located at position $\theta \in \Theta$ implies the
presence of sources at $t = \theta + kT$,~$\forall \, k \in \mathbb Z$.

\subsection{Sensor configuration}

Sensors are assumed to be uniformly distributed over the domain, i.e., if there
are $m$ sensors, they are placed at locations $t = 0$, $T/m$, $2T/m$,~\dots,
$(m{-}1)T/m$ (the offset of the first sensor is arbitrary, so without loss of
generality we take it to be 0). The periodicity of the space ensures that we
automatically also have sensors at $T$, $T{+}T/m$,~\dots\@ The lower bounds we
provide, therefore, are for \emph{this specific sensor configuration}. For a
discussion on why this configuration might be an appropriate choice in the
minimax setting, and on non-uniform sensor placement, see
section~\ref{sec:discussion}.

\subsection{Signal model}
\label{sec:signal-model}

All signals on the aforementioned circular domain are of the form
$f:\Theta\mapsto\mathbb{R}$.  The point source located at $\theta$ is
represented by the impulse signal, $f(t;\theta) = \delta(t - \theta)$, where
$\delta(\cdot)$ is the Dirac delta function.  The sensors observe this signal
through the diffusive medium, which, intuitively speaking, blurs the impulse.
More concretely, we assume that the medium is linear and shift-invariant, so
that it can be represented by a spatial impulse response (blurring would then
correspond to low-pass filtering). Let this impulse response be given by
$g(t)$. Then, the noiseless, continuous-space signal (post-filtering and
pre-sampling) is given by the convolution, $x(t; \theta) = (g*f)(t) = g(t -
\theta)$. Here, we further make the simplifying assumption that $g(t)$ has
restricted support, so that aliasing effects are avoided, and $x(t; \theta)$ is
always well-defined. To be precise, $g(t) = 0$ when $\abs{t} > w/2$, where the
``width'' $w$ of the impulse response satisfies $w < T / 2$ (the reason for the
factor of $1/2$ will become clear in a later section).

The sensors sample this continuous-space shifted impulse response, with some
additive noise. We denote the noiseless sampled version of $x(t; \theta)$ by
the $m$-length vector $\v x(\theta)$:
\begin{equation} \label{eq:sampled-signal}
	\v x(\theta) = \bigg[x(0; \theta), \ldots, x\Big(\frac{kT}{m}; \theta\Big), \ldots, x\Big(\frac{(m-1)T}{m}; \theta\Big)\bigg]^T
\end{equation}
where $m$ is the number of sensors and $k \in \{0, 1, \ldots, m-1\}$. The
additive noise is given by $\v \epsilon$ (to be described shortly), and the
noisy samples are represented by $\v y$:
\begin{equation} \label{eq:sensor-obs}
	\v y = \v x(\theta) + \v \epsilon
\end{equation}
The complete signal model described in this section is summarized in
Fig.~\ref{fig:signal-model}.

\subsection{Channel model}
\label{sec:channel-model}

The noise $\v \epsilon$, introduced in equation~\eqref{eq:sensor-obs}, is
the only source of uncertainty in the problem. For our purposes, we assume that
$\v \epsilon$ is zero-mean and Gaussian, $\v \epsilon \sim \mathcal{N}(\v 0, \m
\Sigma)$. Two kinds of covariance matrices are of particular interest. The
first is dubbed the ``sensor noise'' setting, wherein $\m\Sigma = \sigma^2 \m
I$, i.e. the sensors are afflicted by i.i.d.\ noise. The second is called the
``source noise'' setting, wherein $\m\Sigma$ has off-diagonal terms.
\textcolor{red}{Insert description of source noise.}

With the addition of Gaussian noise, each possible source location $\theta$
gives rise to a different distribution at the sensors, denoted by
\begin{equation} \label{eq:p-theta}
	P(\theta) = \mathcal{N}(\v x(\theta), \m \Sigma).
\end{equation}
$\v y$ is therefore one sample from $P(\theta)$. The space of distributions
produced by all possible source locations is $\mathcal{P} = \{P(\theta) :
\theta \in \Theta \}$. We are interested in computing lower bounds for the loss
function $\Phi(\rho(\theta, \hat\theta)) = \abs{\theta - \hat\theta}^2$, where
$\hat\theta$ is some estimate of the location based on $n$ trials (i.e. $n$
realizations) of the noisy sensor observations $\v y$.

\begin{figure}[tp] %  figure placement: here, top, bottom, or page
	\centering
	\includegraphics[width=3.5in]{block-diagram}
	\caption{A diagramatic representation of the signal model described in
	section~\ref{sec:signal-model}.}
	\label{fig:signal-model}
\end{figure}

\section{Minimax lower bounds for the one-dimensional model}

\subsection{Preliminaries}

We follow the excellent tutorial given in~\cite{Duchi2015Information} to
outline the preliminary steps in deriving bounds on the minimax risk.

Consider the following estimation problem: we have $n$ i.i.d.\ random samples
$Y^n$ from a distribution $P$, which is indexed by a parameter $\theta \in
\Theta$.  Denote the set of these distributions by $\mathcal{P} = \{P(\theta) :
\theta \in \Theta\}$. Suppose we now wish to estimate $\theta$ from $Y^n$.
Define a loss metric $\rho(\theta, \hat\theta)$. For this metric, we can define
the minimax risk over all possible estimators $\hat\theta(Y^n)$ and all
possible $\theta \in \Theta$:
\begin{equation} \label{eq:minimax-expr}
	\mathfrak{M}_n(\mathcal{P}, \Phiorho) = \inf_{\hat\theta} \sup_{\theta \in \Theta} \mathbb E[\Phiorho (\hat\theta(Y^n), \theta)]
\end{equation}
where $\Phi$ is any non-decreasing function (e.g. $\Phi(\rho) = \rho^2$) and
$Y^n$ is the random vector corresponding to a noisy realization of sensor
values.

We start by lower bounding the minimax risk of the estimation problem with the
risk incurred in a multiple hypothesis testing problem. For this, we first need
to define a $2\delta$-packing:%
\begin{definition}
	A set $\Theta_{\mathcal{V}} = \{ \theta_v : v \in \mathcal{V} \}$ for some
	finite index set $\mathcal{V} \subset \mathbb N$ is said to be a
	$2\delta$-packing in the $\rho$-metric if $\rho(\theta_i, \theta_j) \geq
	2\delta \, \forall \, \theta_i, \theta_j \in \Theta_{\mathcal{V}}$.
\end{definition}
\begin{theorem} \label{thm:est-to-testing}%
	If we can find a $2\delta$-packing $\Theta_{\mathcal{V}}$ of $\Theta$, then
	we can lower bound the minimax estimation risk by the average testing risk:
	\begin{equation}
		\mathfrak{M}_n(\mathcal{P}, \Phiorho) \geq \Phi(\delta) \inf_\psi \mathbb P (\psi(Y^n) \neq V)
	\end{equation}
	where $V$ is the unknown, true hypothesis, and $\psi$ is our estimate of
	the hypothesis.
\end{theorem}
For a proof of this theorem, we refer the reader to Proposition~13.3
in~\cite{Duchi2015Information}.  Intuitively, Theorem~\ref{thm:est-to-testing}
says that the error in estimating $\theta_i$ is likely to be more if it is
difficult to distinguish $\theta_i$ from $\theta_j$, i.e.\ if the probability
of error is high (where $\theta_j$ has been selected to be suitably close).

We now need to lower bound the probability of error in the hypothesis testing
problem. The simplest way to do this is to consider a binary hypothesis test
($\abs{\mathcal{V}} = 2$) and use what is known as Le Cam's method:
\begin{theorem} \label{thm:le-cam}
	For a binary hypothesis test, i.e., $\mathcal{V} = \{0, 1\}$,
	\begin{equation}
		\inf_\psi \mathbb P(\psi(Y^n) \neq V) = 1 - \norm{P_0 - P_1}_{TV}
	\end{equation}
	where $P_i$ is short-hand for $P(\theta_i)$ and $\norm{P_0 - P_1}_{TV}$ is
	the total variation distance between the two distributions, defined as
	$\norm{\cdot}_{TV} = \frac{1}{2} \norm{\cdot}_1$.
\end{theorem}
For a proof, we refer the reader to Proposition~2.11
in~\cite{Duchi2015Information}.  Intuitively, Theorem~\ref{thm:le-cam} states
that the minimum probability of error that any estimator must make in a binary
hypothesis testing problem is related to the distance between the distributions
corresponding to the two hypotheses. The closer the two distributions, the
higher the chance of making an error in distinguishing between them.

Thus, the final lower bound can be written as:
\begin{equation} \label{eq:le-cam-bound}
	\mathfrak{M}_n(\mathcal P, \Phiorho) \geq \frac{\Phi(\delta)}{2} \big(1 - \norm{P_1^n - P_0^n}_{TV}\big)
\end{equation}
The superscripts ``$n$'' remind us that these are $n$-fold product
distributions, since we have $n$ i.i.d. trials used in the estimate.  The
difficulty in Le Cam's method lies in selecting the two hypotheses to trade-off
the effect due to a small value of $\delta$ and a large value of $\norm{P_1^n -
P_0^n}_{TV}$ appropriately, to derive the tightest possible bound.

\subsection{Lower bounds using Le Cam's method}

We now state the main result of this paper.
\begin{theorem}
	For a source localization problem as defined in
	section~\ref{sec:source-localization} with sensor noise and a spatial
	impulse response that is $\kappa$-Lipschitz continuous and has restricted
	support of width $w$, the minimax risk in estimating the location of a
	point source is lower bounded by:
	\begin{equation}
		\mathfrak{M}_n(\mathcal{P}, \Phiorho) \geq \frac{1}{128} \frac{\sigma^2 T}{nm\kappa^2w}
	\end{equation}
\end{theorem}

\begin{IEEEproof}
Starting from equation~\eqref{eq:le-cam-bound} and working on the setting
described in Section~\ref{sec:source-localization}, we proceed to derive the
total variation distance for the distributions of interest. Using Pinsker's
inequality~\cite{Pinsker} and the convenient tensorization of the KL
divergence~\cite{tensorization}, we see that:
\begin{equation} \label{eq:pinsker-tensorization}
	\norm{P_1^n - P_0^n}_{TV} \geq \frac{1}{2} D_{KL}(P_0^n \Vert P_1^n) = \frac{n}{2} D_{KL}(P_0 \Vert P_1)
\end{equation}
For multivariate normal distributions with the same covariance, the KL
divergence is given by
\begin{equation} \label{eq:kl-div-normal}
	D_{KL}(P_0 \Vert P_1) = (\v \mu_0 - \v \mu_1)^T \m \Sigma^{-1} (\v \mu_0 - \v \mu_1)
\end{equation}
where $\v \mu_0$ and $\v \mu_1$ are the means of the two Gaussians~\cite{KLDivNormal}.

For the case of sensor noise, $P_i = \mathcal{N}(\v x(\theta_i), \sigma^2 \m
I)$, as described in section~\ref{sec:channel-model}. Hence, combining
equations~\eqref{eq:le-cam-bound}, \eqref{eq:pinsker-tensorization} and
\eqref{eq:kl-div-normal}, we see that
\begin{equation} \label{eq:lb-after-pinsker}
	\mathfrak{M}_n(\mathcal{P}, \Phiorho) \geq \frac{\delta^2}{2} \Bigg[ 1 - \sqrt{ \frac{n}{2\sigma^2} \norm*{\v x(\theta_0) - \v x(\theta_1)}^2 } \Bigg]
\end{equation}
Let $\v \Delta = \v x(\theta_0) - \v x(\theta_1)$ for brevity. Denoting the
$k$-th component of $\v\Delta$ as $\Delta(k)$, we see that
\begin{align}
	&\norm{\v\Delta}^2 = \sum_{k=0}^{m-1} \abs{\Delta(k)}^2 = \sum_{k=0}^{m-1} \abs{\Delta(k)}^2 \, \mathbb I_{\{\ell: \abs{\Delta(\ell)} > 0\}}(k) \\
	&= \sum_{k=0}^{m-1} \abs[\Big]{x\Big(\frac{kT}{m}; \theta_0\Big) - x\Big(\frac{kT}{m}; \theta_1\Big)}^2 \mathbb I_{\{\ell: \abs{\Delta(\ell)} > 0\}}(k) \\
	&= \sum_{k=0}^{m-1} \abs[\Big]{g\Big(\frac{kT}{m} - \theta_0\Big) - g\Big(\frac{kT}{m} - \theta_1\Big)}^2 \mathbb I_{\{\ell: \abs{\Delta(\ell)} > 0\}}(k)
\end{align}
where $x(t;\theta_i)$ is the continuous-space filtered signal described in
section~\ref{sec:signal-model}. For an impulse response $g$ which is Lipschitz
continuous with parameter $\kappa$, we can upper bound the term within the
summation:
\begin{equation}
	\abs[\Big]{g\Big(\frac{kT}{m} - \theta_0\Big) - g\Big(\frac{kT}{m} - \theta_1\Big)} \leq \kappa \abs{\theta_0 - \theta_1} = \kappa \cdot 2\delta
\end{equation}
since $\theta_1 = \theta_0 + 2\delta$ (by virtue of the $2\delta$ packing).
Hence,
\begin{equation}
	\norm{\v\Delta}^2 \leq 4 \kappa^2 \delta^2 \sum_{k=0}^{m-1} \mathbb I_{\{\ell: \abs{\Delta(\ell)} > 0\}}(k) = 4 \kappa^2 \delta^2 \norm{\v\Delta}_0.
\end{equation}
$\norm{\v\Delta}_0$ counts the number of non-zero elements in $\v\Delta$, which
is equal to the number of sensors in the total region covered by the signals
$x(t;\theta_0)$ and $x(t;\theta_1)$ (see figures~\ref{fig:overlap-middle} and
\ref{fig:delta-sampled}). Therefore, $\norm{\v\Delta}_0 = \floor[\big]{\frac{(w
+ 2\delta)m}{T}}$, since at most a fraction $(w + 2\delta) / T$ of sensors can
lie in the region covered by the two impulse responses. The final upper bound
on $\norm{\v\Delta}^2$ is therefore
\begin{equation} \label{eq:delta-bound}
	\norm{\v\Delta}^2 \leq 4 \kappa^2 \delta^2 \floor*{\frac{(w + 2\delta)m}{T}} \leq 4 \kappa^2 \delta^2 \frac{m (w + 2\delta)}{T}.
\end{equation}
Since sensors are uniformly distributed, and since the domain is periodic, this
holds even if $\theta_0$ lies at the edge of the domain (close to $t=0$, for
e.g.). One part of the signal $x(t;\theta_0)$ will appear at the left edge of
the domain, and the repetition from the period $[T, 2T)$ will appear at the
right edge of the domain. Also note that the two signals $x(t;\theta_0)$ and
$x(t;\theta_1)$ overlap at most once, since $w < T/2$, as stated in
section~\ref{sec:signal-model}.

\begin{figure}[tp] %  figure placement: here, top, bottom, or page
	\centering
	\includegraphics[width=2.8in]{overlap-middle-pics}
	\caption{Depiction of the impulse responses corresponding to two
	hypotheses, $\theta_0$ and $\theta_1$. Note that each impulse response has
a support of size $w$, and the two are separated by a distance $2\delta$. The
total size of the support of the difference signal (subtracting one impulse
response from the other, pointwise) $\Delta(t)$ is therefore at most $w +
2\delta$.}
	\label{fig:overlap-middle}
\end{figure}

The upper bound in equation~\eqref{eq:delta-bound} translates into a lower
bound for equation~\eqref{eq:lb-after-pinsker}:
\begin{align}
	\mathfrak{M}_n(\mathcal{P}, \Phiorho) &\geq \frac{\delta^2}{2} \Bigg[ 1 - \sqrt{\frac{n}{2\sigma^2 T} 4 \kappa^2 \delta^2 (w + 2\delta) m} \Bigg] \\
	&= \frac{\delta^2}{2} \Bigg[ 1 - \sqrt{\frac{2nm \kappa^2 \delta^2 (w + 2\delta)}{\sigma^2 T}} \Bigg] \label{eq:lb-after-delta-bound}
\end{align}

Asymptotically, we choose smaller values of $\delta$ for larger number of
sensors. Hence, neglecting terms of order $\delta^3$, we tighten the bound by
choosing $\delta$ as a function of the remaining variables in order to have
$\sqrt{\frac{2nm\kappa^2\delta^2 }{\sigma^2}} = \frac{1}{2}$. This is achieved
for $\delta = \sqrt{\frac{\sigma^2T}{8nm\kappa^2 w}}$, so that for
$\Phi(\delta) = \delta^2$, equation~\eqref{eq:lb-after-delta-bound} becomes
\begin{equation}
	\mathfrak{M}_n(\mathcal{P}, \Phiorho) \geq \frac{1}{128} \frac{\sigma^2 T}{nm\kappa^2 w}.
\end{equation}
This completes the proof.
\end{IEEEproof}

\section{Discussion}
\label{sec:discussion}

\textcolor{red}{Needs rewrite.}

The reason that the local Fano method gives a poorer bound than Le Cam's method
is that we're using the local Fano method on a bounded domain, making $V$ as a
function of the packing size $\delta$.  It is evident even from examining
equations \eqref{eq:fano-lb} and \eqref{eq:mutual-info-kl-reln} that taking
just two hypotheses with a suitable separation will yield tighter bounds, since
the cardinality of the packing set $V$ will no longer be dependent on $\delta$.
This will allow us to make use of the expression in equation
\eqref{kl-vi-vj-final} to choose $\delta$, which will yield the rate we saw in
Le Cam's method. It is to be seen whether using the global Fano
method~\cite{Duchi2015Information} will improve the rate, or at least, the
constants that we attained with Le Cam's technique. A more rigorous treatment,
which finds an upper bound for the sum, rather than using integrals, so as to
give bounds for even moderate numbers of sensors is also delegated to future
work.

%\IEEEtriggeratref{8}

\bibliographystyle{IEEEtran}
\bibliography{IEEEabrv,references}

\end{document}
